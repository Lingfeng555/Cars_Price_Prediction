\begin{tabular}{rll}
\toprule
C & kernel & gamma \\
\midrule
12.235245 & rbf & scale \\
\bottomrule
\end{tabular}
